
\documentclass[bibend=bibtex,lang=cn]{elegantpaper}

\title{钱一笑二月月报}
\author{钱一笑}
\date{\today}
\institute{浙江大学}
\theoremstyle{plain}
\newtheorem{exercise}{练习}[section]
\newenvironment{solution}{\begin{proof}[\textbf{\emph 解}]}{\end{proof}}
\everymath{\displaystyle}

\begin{document}

\maketitle

\section{二月进度报告}

2023年的2月由于开学期末考试以及3月初的毕业论文开题答辩的影响,大致只有2.15以前在执行科研工作,
以下是2023年2月的简单总结:

\begin{itemize}
\item 完成不规则区域椭圆方程多重网格数值求解的初步实现,在几个测试中保持与原程序相同的输出结果,单机测试下速度也比原程序要快,具体结果打算在进行多机测试后以报告形式上交。
\item BiCGStab的并行实现,当前采用的并行BiCGStab并行算法是\cite{krasnopolsky2010reordered}给出的重排算法,该重排算法通过重叠MPI通讯和预处理操作使得算法更加高效。不过在单机多进程测试情况下还是比直接求解慢了不少,还需要优化以及充分多机测试。
\item 有限元复习与考试:程老师的有限元课程上个学期没好好学,2月花了十天左右复习。程老师上的内容整体更偏向有限元方法的理论,动手实践偏少。说实在学完整门课后,大篇幅的有限元误差分析没学会,
  实际能上手编程求解的方程也没几个,所以感觉学到的内容不多。
\item 撰写开题报告和开题答辩幻灯片:2.21考完有限元后就加紧写开题答辩幻灯片和开题报告,基本上写完这两个2月也就结束了。
\end{itemize}

\section{下一步计划}

\subsection{机房软硬件配置与程序测试}

最近已经和梁凯毅学长开始规划机房的软硬件部署,打算先部署统一身份验证NIS和网络文件系统NFS,
然后部署并行调度软件slurm,这样就可以运行测试当前的并行程序了。
也想问一下张老师还有无部署Raid阵列的需求,如果有需求我们也准备去部署下。

\subsection{不规则区域算法负载均衡}

最近读了一些并行文献考虑进程之间的负载均衡问题,
大多数规则区域的求解方法的论文中都是采用平均划分定义域,没有什么参考价值。
而针对不规则区域,\cite{adelmann2010fast}中使用了所谓的「递归坐标二分」(recursive coordinate bisection)来给每个进程分配计算域,
我还没仔细研究,不过看上去感觉类似于OpenMP中进程的动态调度算法。
我当前的想法是:如果不希望调度算法做得非常复杂,
那么还是可以使用静态调度的方案,
基本想法是在输入文件中指定定义域划分的基本粒度,
分别给予不规则控制体和规则控制体不同的权重,
按照权重将计算域分配给各个进程,具体实现还需要与颜嘉图学长商量商量。

\subsection{多重网格算法的改进}

最近在读文献的过程中,发现劳伦斯伯克利国家实验室他们已经对多重网格算法的可并行性做了充分的研究,
\cite{chow2006survey}中包含了许多并行多重网格的思路,不过2月还没来得及仔细研究,打算3月好好研究一下他们的并行策略。
我粗略浏览了一下,感觉现在比较需要的可能是并行化分块LU分解,
原因是当前BiCGStab的速度暂时跟不上LU分解的速度,
而计算规模较大时可能无法将整个LU分解流程放到单机上运行。

\subsection{时间安排}

本学期我就选了计算机系统3(操作系统+计算机体系结构)一门课,和王廷源合作担任王何宇老师课程助教,
因此空余时间比较多,可以快速推进工作。
我感觉基本能够在三月份完成不规则区域椭圆方程并行程序的所有工作,
也希望张老师指示一下后续的工作规划。


\nocite{*}
\printbibliography[heading=bibintoc, title=\ebibname]


\end{document}


